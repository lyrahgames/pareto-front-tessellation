\documentclass{stdlocal}
\begin{document}
\section{Problem} % (fold)
\label{sec:problem}
  \subsection{Input} % (fold)
  \label{sub:input}
    A discrete set of points $\mathcal{S}$ in $\realnumbers^n$ with $n\in\naturalnumbers$ is given.
    For our purposes, $\mathcal{S}$ will be the result of a multi-objective optimization in $n$ dimensions.
    Hence, most of the given points will form the Pareto frontier.
  % subsection input (end)

  \subsection{Output} % (fold)
  \label{sub:output}
    Based on the given point set $\mathcal{S}$, a triangulation or tessellation shall be computed which enables the user to securely interpolate between Pareto points, to visualize the resulting surfaces, and to localize new points fast for configuration.
    The typical problem that has to be taken care of is that Pareto frontiers are not continuous in general.
    Furthermore, multi-objective optimization tends to use many dimensions which are difficult to visualize.
  % subsection output (end)

  \subsection{Solution} % (fold)
  \label{sub:solution}
    A simple solution would be to ignore any kind of surface construction and use the sticky point algorithm.
    But looking at visualization and configuration this is suboptimal.

    \begin{enumerate}
      \item Compute Pareto points.
      \item Project Pareto points from $n$-dimensional space to the $n-1$-dimensional hyperplane with respective normal in direction $\sum_{i=1}^n e_i$.
      \item Construct the $n-1$-dimensional Delaunay triangulation/tessellation of the projected points and go back to $n$ dimensions.
      \item Statistically analyze the point distances and construct an approximating probability distribution for distances of neighboring points.
      \item Make hypothesis tests and remove edges/facets and their adjacent triangles/simplices to get rid of non-continuous areas.
      \item Visualize the results by projecting the data to two- and three-dimensional space and rendering it.
      \item Use the structures for configuration by enabling fast localization of points inside the triangulation/tessellation.
    \end{enumerate}
  % subsection solution (end)
% section problem (end)
\end{document}