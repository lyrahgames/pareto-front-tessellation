\documentclass{stdlocal}
\begin{document}
\section{Problem} % (fold)
\label{sec:problem}
  \subsection{Input} % (fold)
  \label{sub:input}
    A discrete set of points $\mathcal{S}$ in $\realnumbers^n$ with $n\in\naturalnumbers$ is given.
    For our purposes, $\mathcal{S}$ will be the result of a multi-objective optimization in $n$ dimensions.
    Hence, most of the given points will form the Pareto frontier.
  % subsection input (end)

  \subsection{Output} % (fold)
  \label{sub:output}
    Based on the given point set $\mathcal{S}$, a triangulation or tessellation shall be computed which enables the user to securely interpolate between Pareto points, to visualize the resulting surfaces, and to localize new points fast for configuration.
    The typical problem that has to be taken care of is that Pareto frontiers are not continuous in general.
    Furthermore, multi-objective optimization tends to use many dimensions which are difficult to visualize.
  % subsection output (end)

  \subsection{Solution} % (fold)
  \label{sub:solution}
    A simple solution would be to ignore any kind of surface construction and use the sticky point algorithm.
    But looking at visualization and configuration this is suboptimal.

    \begin{enumerate}
      \item Estimate a sufficient amount of points near the actual Pareto frontier by using an appropriate multiobjective problem solver, like NSGA-II \autocite{deb2002}, which uses the crowding distance metric or an alternative similar method to provide approximately uniformly distributed points on the Pareto frontier.

      \item Project the estimated points from $m$-dimensional objective vector space to the $m-1$-dimensional hyperplane with respective normal in direction $\sum_{i=1}^m e_i$.
      Because every estimated point is not dominated by other points, the transformation is invertible and makes Delaunay tessellation of the curved Pareto surface possible.

      \item Construct the $m-1$-dimensional Delaunay triangulation/tessellation of the projected points by using a state-of-the-art algorithm and apply the same tessellation to the original points in $m$-dimensional space.
      Inverting the projection is not needed.
      It is recommended to use a facet/edge-based data structure, like the quad-edge structure from \textcite{guibas1985} in two dimensions, to store the tessellation scheme.
      Such a structure has to be general enough to describe arbitrary connections between points even if there is no triangle.
      Such a structure is easy to store and transmit in binary- and ASCII-based file formats.
      Following post computation will also become easier and triangles can be constructed by iterating through all connections.

      \item Statistically analyze the distances of points to their neighbors and the estimated gradient at their position.
      This can be done in a global sense for all points or for smaller batches of points.

      \item Use a heuristic to combine those two values. Construct an approximating probability distribution for such a heuristic and learn it.

      \item Make hypothesis tests based on the learned probability distribution and remove point connections from the tessellation that do not fulfill the test.
      They could exhibit discontinuities.

      \item Visualize the results by projecting the data to two- and three-dimensional space and rendering it.

      \item Use the Delaunay tessellation structure for configuration by using fast localization of points \autocite{guibas1985}.
    \end{enumerate}
  % subsection solution (end)
% section problem (end)
\end{document}